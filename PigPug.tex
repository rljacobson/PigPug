\documentclass[letterpaper]{report}
%\documentclass[a4paper]{article}

\usepackage{amsmath,amssymb}
\usepackage[]{microtype}
\usepackage{enumitem}
\setlist{nosep}

\usepackage[linguistics]{forest}
%\usetikzlibrary{arrows.meta}

\forestset{
	default preamble={
		for tree={
%			grow=east,
%			parent anchor=south,
%			child anchor=west,
			align=left,
%			s sep+=1cm,
%			inner sep=2pt,
			edge={->}, % Makes the edges arrows
			calign = first,
		}
	}
}


\newcommand{\mgsu}{
	{\sc mgsu}%
}


% Title Page
\title{}
\author{}


\begin{document}
%\maketitle


\section*{Building-in Equational
	Theories}\label{chapter-building-in-equational-theories}

G. D. Plotkin\\
\emph{Department of Machine Intelligence}\\
\emph{University of Edinburgh}

\subsection*{Introduction}\label{introduction}

If let loose, resolution theorem-provers can waste time in many ways.
They can continually rearrange the multiplication brackets of an
associative multiplication operation or replace terms $t$ by ones like
$f(f(f(t,e),e),e)$ where $f$ is a multiplication function and $e$
is its identity. Generally they continually discover and misapply
trivial lemmas. Global heuristics using term complexity do not help much
and \emph{ad hoc} devices seem suspicious.

On the other hand, one would like to evaluate terms when possible, for
example we would want to replace $5+4$ by $9$. More generally one
would like to have liberty to simplify, to factorise and to rearrange
terms. The obvious way to deal with an associative multiplication would
be to imitate people, and just drop the multiplication brackets. However
used or abused the basic facts involved in such manipulations form an
equational theory, $\mathrm{T}$, that is, a theory all of whose sentences are
universal closures of equations.

Under certain conditions, we will be able to build the equational theory
into the rules of inference. The resulting method will be
resolution-like, the difference being that concepts are defined using
provable equality between terms rather than literal identity. Therefore
the set of clauses expressing the theory will not be among the input
clauses, so no time will be wasted in the misapplication of trivial
lemmas, since the rules will not waste time in this way.

Such devices as evaluation, factorisation and simplification consist of
the application of a recursive function, $N$, from terms to terms with
the property:
\[
\vdash_{\mathrm{T}}t=N(t).
\]
For example, given an associative function $f$ in $t$, $N$ might
rearrange $t$ with all the $f$'s as far to the right
as possible; for an arithmetic expression $t$, $N(t)$ might be the
value.

The simplification function can be extended to one from literals and
clauses to, respectively, literals and clauses by:
\begin{align*}
	&N((\pm)P(t_{1},\ \cdots,\ t_{n})) = (\pm)P(N(t_{1}),\ \cdots,\ N(t_{n}))\\
	&N(C)=\{N(L)\mid L\in C\}
\end{align*}
and similarly to other expressions.

We shall look for a complete set of rules $r_{1},\ \cdots,\ r_{m}$
such that $r_{1}\circ N,\ \cdots,\ r_{m}\circ N$ is also a complete
set. (If $r$ is a rule, $r\circ N$ is the rule which outputs
$N(C)$ iff $r$ outputs, with the same inputs, $C$). So the
required facilities can be used, but incorporated in a theoretical
framework.

While one could just drop the brackets of an associative operation, and
make appropriate changes, such a procedure could not be systematic.
Instead note that the set of terms in right associative normal form is
in a natural 1-1 correspondence with the set of
`terms' in which the multiplication
brackets and symbols, and the commas are dropped. So the bracketless
terms will be considered simply as an abbreviation for the normal form.
This situation seems to be quite common. Another example: if $\mathrm{T}$ is
the theory of Boolean algebra, terms can be represented as sets of sets
of literals, being in 1-1 correspondence, according to convention, with
terms in either a suitable disjunctive or conjunctive normal form.

Normal forms will be considered to be given by a simplification
function, $N$, which in addition satisfies the postulate:

\begin{quote}
If $\vdash_{\mathrm{T}}t=u$ then $N(t)$ is identical to
$N(u)$.
\end{quote}


A theory $\mathrm{T}$ has a normal form iff it is decidable.

This does not cover all uses of the phrase. For example in the untyped
$\lambda$-calculus, not all terms have one and $N$ is only partial
recursive. And of course things other than terms can have normal forms.
As defined, normal form functions give less redundancy than any other
simplification function.

The special characteristic of resolution-like methods is the unification
algorithm. In our case, unification is defined relative to the theory
$\mathrm{T}$. A substitution, $\sigma$, $\mathrm{T}$-unifies two terms $t$ and
$u$ iff $\vdash_{\mathrm{T}}t\sigma=u\sigma$. The exact
generalisation of the notion of most general unifier will be given
later; in general rather than a single most general unifier, there will
be an infinite set of maximally general ones.

Define an $N$-application and composition operator, $*$, by:
\begin{align*}
&t*\sigma=N(t\sigma)\\
&\sigma*\mu=N(\sigma\mu)
\end{align*}

It can be shown that $*$ possesses properties analogous to ordinary
application and composition independently of the particular $\mathrm{T}$ and
$N$ chosen (see Robinson 1965).

For each equational theory one must invent a special unification
algorithm with these equations built in. If we know a normalising
function for the theory we are likely to get a more efficient algorithm.
The purpose of this paper is not to display such algorithms, although we
will give a couple of examples to fix ideas. Our purpose is to show how,
given any such unification algorithm satisfying certain conditions, the
usual resolution techniques can
be carried through using the more sophisticated algorithm, and without
losing completeness.

Now as an example, consider the case where $\mathrm{T}$ is the theory of an
associative function, $f$, whose language may include other function
symbols, and $N(t)$ is the right associative form of $t$.

How can we unify $f(x,y)$ with $f(a,f(b,c))$ which is equal to
$f(f(a,b),c)$? There are two $\mathrm{T}$-unifications: $x = a$, $y = f(b,c)$
and $x = f(a,b)$, $y = c$. The first is straightforward. The second
can be obtained in two steps:

\begin{enumerate}
	\def\labelenumi{(\arabic{enumi})}
	\item
	Put $x = f(a,x')$ where $x'$ is a new variable.
	\item
	Unify $f(f(a,x'),y)$ with $f(a,f(b,c))$.
\end{enumerate}

In step (2) we should normalise both expressions before unifying, so we
actually unify $f(a,f(x',y))$ with $f(a,f(b,c))$. Thus
$x' = b$ and $y = c$, giving us the second unification.

In practice it is simpler to reuse the variable $x$ instead of
introducing a new variable $x'$.

The unification algorithm is non-deterministic; a substitution is in the
maximally general set of $\mathrm{T}$-unifiers it defines iff the substitution
can be output by the algorithm. That this set has the correct
properties, and the correctness of the other algorithm given in this
introduction is proved in Plotkin (1972). Terms are, temporarily,
regarded as strings of function symbols, brackets, commas and variables.
The disagreement pair of two terms $t$ and $u$ is the rightmost pair
of distinct terms $t'$ and $u'$, such that $t$ and
$u$ have the forms $Xt'Y$ and $Xu'Z$ respectively,
for suitable strings $X, Y$ and $Z$.

\begin{figure}[h]
%	\centering
\begin{forest} 
[
	{$f(x, z)$\\ $f(a, f(y, b))$}
	[
		{$f(a, z)$\\ $f(a, f(y, b))$}
		[
			{$f(a, f(y, b))$\\ $f(a, f(y, b))$}
			[\it success]
		]
	]
	[
		{$f(a, f(x, z))$\\ $f(a, f(y, b))$}
		[
			{$f(a, f(y, z))$\\ $f(a, f(y, b))$}
			[
				{$f(a, f(y, b))$\\ $f(a, f(y, b))$}
				[\it success]
			]
		]
		[
			{$f(a, f(y, f(x, z)))$\\ $f(a, f(y, b))$}
			[\it failure]
		]
		[
			{$f(a, f(x, z))$\\ $f(a, f(x, f(y, b)))$}
			[
				{$f(a, f(x, f(y, b)))$\\ $f(a, f(x, f(y, b)))$}
				[\it success]
			]
		]
	]
]
\end{forest}\\
\small Figure 1
\end{figure}


\begin{enumerate}
	\def\labelenumi{(\arabic{enumi})}
	\item
	Set $\sigma$ equal to $\varepsilon$.
	\item
	If $t$ is identical to $u$, stop with success and output
	$\sigma$.
	\item
	Find the disagreement pair, $t'$, $u'$ of $t$
	and $u$.
	\item
	If $t'$ and $u'$ begin with different function
	symbols, stop with failure.
	\item
	If $t'$ and $u'$ are both variables, change $t$,
	$u$ and $\sigma$ to $t*\lambda$, $u*\lambda$ and
	$\sigma*\lambda$ where $\lambda$ is either
	$\{u'/t'\}$,
	$\{f(u',t')/t'\}$ or
	$\{f(t',u')/u'\}$.
	\item
	If $t'$ is a variable and $u'$ is not, then, if
	$t'$ occurs in $u'$, stop with failure; otherwise
	change $t$, $u$ and $\sigma$ to $t*\lambda$, $u*\lambda$ and
	$\sigma*\lambda$ where $\lambda$ is either
	$\{u'/t'\}$ or
	$\{f(u',t')/t'\}$.
	\item
	If $u'$ is a variable and $t'$ is not, then, if
	$u'$ occurs in $t'$, stop with failure; otherwise
	change $t$, $u$ and $\sigma$ to $t*\lambda$, $u*\lambda$ and
	$\sigma*\lambda$ where $\lambda$ is either
	$\{t'/u'\}$, or
	$\{f(t',u')/u'\}$.
	\item
	Go to 2.
\end{enumerate}

Figure 1 gives an example which traces the values of $t$ and $u$
through the execution tree of the algorithm when $t =  f(x,z)$ and
$u =  f(a,f(y,b))$, giving all the successful and some of the
unsuccessful searches.

The set of unifiers is
$\{\{a/x, f(y,b)/z\}$, $\{f(a,y)/x,\,b/z\}$, $\{f(a,x)/x$, $f(x,y)/y$, $f(y,b)/z\}\}$.

One can have infinitely many successes, as illustrated in figure 2,
where we are now using the abbreviation mechanism, writing $xy$ for
$f(x,y)$; $g$ is some other function. In this case, the set of
unifiers produced is
\begin{quote}
$\{\{a^{n}/x,\,a^{n}/y\}\mid \,n  >  0\}.$
\end{quote}

Infinite failures are also possible: for example, take
\begin{quote}
	$t =  g(x, xa)$ and $u =  g(y, by)$.
\end{quote}


\begin{figure}[h]
\begin{forest} 
	[
		{$g(x, xa)$\\ $g(y, ay)$}
		[
			{$g(y, ya)$\\ $g(y, ay)$}, name=wilma
			[
				{$g(a, aa)$\\ $g(a, aa)$}
				[\it success]
			]
			[
				{$g(ay, a ya)$\\ $g(ay, a ay)$},
				name=barny
			]
		]
		[
			{$g(yx, yxa)$\\ $g(y, ay)$}
			[\it failure]
		]
		[
			{$g(x, xa)$\\ $g(xy, axy)$}
			[\it failure]
		]
	]
	\draw[->, dashed] (barny) to[out=90, in=0] (wilma);
\end{forest}\\
\small Figure 2
\end{figure}

We conjecture that it is possible to decide whether two terms have a
unifier -- this is essentially the same problem as deciding whether a
set of simultaneous equations in a free semigroup has a solution.

As a second example, we give an `arithmetical
evaluation' theory. Among its function symbols are two
binary ones $+$ and $\times$ and infinitely many constants -- the
numerals $\Delta_{n}(n \geqslant 0)$. The axioms are:

\begin{enumerate}
	\def\labelenumi{(\arabic{enumi})}
	\item
	$+(\Delta_{m},\,\Delta_{n}) = \Delta_{m+n}$ \quad($m,\,n \geqslant 0$)
	\item
	$\times(\Delta_{m},\,\Delta_{n}) = \Delta_{m\times n}$
	\quad($m,\,n \geqslant 0$).
\end{enumerate}
These are just the addition and multiplication tables.

$N(t)$ is obtained from $t$ by continual application of the
reduction rules:

\begin{enumerate}
	\def\labelenumi{(\arabic{enumi})}
	\item
	Replace an occurrence of $+(\Delta_{m},\,\Delta_{n})$ by one of
	$\Delta_{m+n}$ \quad($m,\,n \geqslant 0$)
	\item
	Replace an occurrence of $\times(\Delta_{m},\,\Delta_{n})$ by one of
	$\Delta_{m\times n}$ \quad($m,\,n \geqslant 0$)
\end{enumerate}

A term, $t$, is a \emph{number} term iff the only function symbols
occurring in $t$ are $+,\,\times$ or numerals.

Suppose Eq is a set of equations and Ineq is a set of sets of
inequations and all terms in Eq or Ineq are number terms. A
substitution, $\sigma$, \emph{solves} Eq and Ineq iff:

\begin{enumerate}
	\def\labelenumi{(\arabic{enumi})}
	\item
	If $x$ is a variable in Eq and Ineq then $x\sigma$ is a numeral
	and otherwise $x\sigma$ is $x$.
	\item
	If $t = u$ is in Eq then $t*\sigma$ is identical to $u*\sigma$.
	\item
	Every member of Ineq has a member, $t \neq u$, such that
	$t*\sigma$ is distinct from $u*\sigma$.
\end{enumerate}

For example, $\{3/x\}$ solves ($\{x + 2 = 5\}$,
$\{\{x \neq 0\}\}$).

In general $\{\sigma\mid \sigma\,\mbox{solves Eq and Ineq}\,\}$ is a
partial recursive set, being the set of solutions of a certain set of
Diophantine equations and inequations, effectively obtainable from Eq
and Ineq.\\
The algorithm for producing a maximally general set of unifiers for two
normal terms $t$ and $u$ is:

\begin{enumerate}
	\def\labelenumi{(\arabic{enumi})}
	\item
	Set Unify to $\{t = u\}$, Eq and Ineq to $\varnothing$ and
	$\sigma$ to $\varepsilon$.
	\item
	Remove equations of the form $t = t$ from Unify.
	\item
	If Unify is empty, let $\mu$ be a substitution which solves Eq and
	Ineq, change $\sigma$ to $\sigma*\mu$ and stop with success.
	\item
	Remove an equation, $t = u$, from Unify.
	\item
	If $t$ and $u$ are variables, replace Unify, Eq, Ineq and
	$\sigma$ by Unify $*\lambda$, Eq $*\lambda$, Ineq $*\lambda$
	and $\sigma*\lambda$ respectively, where $\lambda = \{t/u\}$.
	\item
	Suppose $t$ is a variable and $u$ is a number term. If $t$
	occurs in $u$, put $t = u$ in Eq; otherwise replace Unify, Eq,
	Ineq and $\sigma$ by $\mathrm{Unify}*\lambda$, $\mathrm{Eq}*\lambda$, $\mathrm{Ineq}*\lambda$ and $\sigma*\lambda$ respectively where
	$\lambda = \{u/t\}$.
	\item
	Suppose $t$ is a variable and $u$ is a term, but not a number one.
	If $t$ occurs in $u$, stop with failure; otherwise replace Unify,
	Eq, Ineq and $\sigma$ by $\mathrm{Unify}*\lambda$, $\mathrm{Eq}*\lambda$, $\mathrm{Ineq}*\lambda$ and $\sigma*\lambda$ respectively, where
	$\lambda = \{u/t\}$.
	\item
	If $u$ is a variable and $t$ is not, proceed as in steps 6 and 7,
	but reverse the roles of $t$ and $u$.
	\item
	If $t$ and $u$ begin with distinct function symbols then if either
	of them are not number terms stop with failure, otherwise put
	$t = u$ in Eq.
	\item
	If $t$ and $u$ begin with the same function symbol, $f$, say,
	and one of them is not a number term, then add 
	\begin{quote}
	$t_{1} = u_{1}$,
	$\ldots$, $t_{n} = u_{n}$ to Unify where $t = f(t_{1}$,
	$\ldots$, $t_{n})$ and $u = f(u_{1}$, $\ldots$, $u_{n})$.
	\end{quote}
	\item
	If $t$ and $u$ have the forms $g\left(t_{1},t_{2}\right)$ and
	$g\left(u_{1},u_{2}\right)$ respectively and both are number terms,
	where $g$ is either $+$ or $\times$, then either add $t = u$
	to Eq and $\{t_{1} \neq u_{1},\,t_{2} \neq u_{2}\}$ to Ineq or else
	add $t_{1} = u_{1}$ and $t_{2} = u_{2}$ to Unify.
	\item
	Go to 2.
\end{enumerate}

In the above, steps 3 and 11 are non-deterministic. The equation removed
in step 4 is to be chosen in some fixed way. The algorithm always
terminates, and can be implemented efficiently (apart from the
difficulty of having to solve arbitrary Diophantine equations!). It uses
no special properties of $+$ or $\times$ and so can be adjusted to
suit any similar evaluation theory.

For example, suppose $t$ is $\times(x,x)$ and $u$ is
$\times(\Delta_{4},\,\times(y,y))$, then, at the beginning of the
algorithm, we have
$\mbox{Unify} = \{\times(x,x) = \times(\Delta_{4},\,\times(y,y))\}$,
Eq$ = \mbox{Ineq} = \mbox{$\varnothing$}$ and
$\sigma = \varepsilon$. Next, the execution sequence splits at step
(11). Either Unify becomes $\varnothing$, Eq becomes
$\{\times(x,x) = \times(\Delta_{4},\,\times(y,y))\}$, Ineq becomes
$\{\{x \neq \Delta_{4},\,x \neq \times(y,y)\}\}$ and $\sigma$ still
has the value $\varepsilon$, or else Unify becomes
$\{x = \Delta_{4},\,x = \times(y,y)\}$, Eq and Ineq remain at
$\varnothing$ and $\sigma$ at $\varepsilon$. In the first case,
the algorithm stops successfully at step (3), via a $\mu$ of the form
$\{\Delta_{2n}/x,\,\Delta_{n}/y\}$, where $n \geqslant 0$ and
$n \neq 2$, and then $\sigma = \mu$. In the second case, supposing
$x = \Delta_{4}$ to be chosen at step (2), after step (6) we have
$\mbox{Unify} = \{\Delta_{4} = \times(y,y)\}$,
Eq$ = \mbox{Ineq} = \mbox{$\varnothing$}$ and
$\sigma = \{\Delta_{4}/x\}$. After steps (9) and (3), this execution
path terminates with $\sigma = \{\Delta_{4}/x,\,\Delta_{2}/y\}$. If
the commutativity and associativity of $\times$ were also built-in,
one would expect here the single result, $x = \times(\Delta_{2},y)$.

As a final example, the theory of an associative, commutative,
idempotent binary function also has a unification algorithm -\/-
although we only know an extremely inefficient one. In this theory every
pair of terms has a finite maximally general set of unifiers. This can
be considered as building in sets, to some extent.

We believe most common algebraic systems admit unification algorithms.
In particular we believe one can build in bags, lists and tuples in this
way (Rulifson 1970). Group theory seems difficult. In general, the
problem of finding a maximally general set of unifiers resembles, but is
easier than the problem of solving in a closed form, equations in the
corresponding free algebra. Other.people have designed unification
algorithms (Bennett, Easton, Guard and Settle 1967, Gould 1966, Nevins
1971, Pietrzykowski 1971, Pietrzykowski and Jensen 1972) but have not
demonstrated that their methods satisfy the independence condition or,
with the exception of Pietrzykowski \emph{et al.}, the completeness one
(see later). Cook (1965) seems to be the first person to notice the
usefulness of normal forms in the theorem-proving context.

\subsection*{Formal preliminaries}\label{formal-preliminaries}

The formalism is that of Robinson (1965) and Schoenfield (1967). The two
are compatible with some minor adjustments. Clauses should be regarded
as abbreviations for a corresponding sentence. Schoenfield omits
brackets in his terms; these should be put back.

We use the equality symbol in infix mode and regard $t = u$ and
$u = t$ as indistinguishable, thus building-in symmetry. Although the
logic is not sorted, only minor adjustments are necessary to accommodate
disjoint sorts.

The letters $t$, $u$, $\ldots$ range over terms. An occurrence of
$t$ in $v$ is indicated by $v(t)$; $v(u/t)$ indicates the term
obtained from $v(t)$ by replacing that distinguished occurrence of
$t$ in $v$ by one of $u$.

The letters $L$, $M$, $\ldots$ range over literals. The general
form of a literal is $(\pm)$$P(t_{1},\ldots,t_{n})$. $\bar{L}$ is
like $L$, but has opposite sign.

The letters $C$, $D$, $\ldots$ range over clauses. By convention,
$C \lor L$ represents the clause $C\cup\{L\}$ and implies $L$ is
not in $C$. Similarly $C\lor D$ represents $C\cup D$, and implies
$C\cap D = \varnothing$ and $D \neq \varnothing$. Equations are unit
clauses, $\{t = u\}$; $\{t \neq u\}$ is the general form of an
inequation. The clause $\overline{C}$ is $\{\bar{L}\,\mid L$ in
$C\}$.

The letter $S$ varies over sets of clauses and Eq varies over sets of
equations. The letter $\mathcal{E}$ stands for a set of sets, each
member of which is either a finite set of terms or a finite set of
literals. Greek letters $\sigma$, $\mu$, $\ldots$ range over
substitutions: $\varepsilon$ is the empty substitution. If
$\xi = \{y_{1}/x_{1},\ldots,y_{n}/x_{n}\}$ and the $y_{i}$ are all
distinct then $\xi$ is \emph{invertible} and $\xi^{-1}$ is defined
to be $\{x_{1}/y_{1},\ldots,x_{n}/y_{n}\}$. The letter $\xi$ is
reserved for invertible substitutions. With each pair of clauses $C$
and $D$, an invertible substitution $\xi_{C,D}$ is associated in
some standard way, and is such that $C\xi_{C,D}$ and $D$ have no
common variables and if $x$ does not occur in $C$, $x\xi_{C,D}$ is
$x$.

The letter $V$ ranges over finite sets of variables. Var(---) is
the set of variables in the syntactic entity ---, which can be a
term, literal, clause or sets of sets of $\ldots$ such; substitutions
may be applied to such entities in the natural fashion. The
\emph{restriction} of $\sigma$ to $V$, $\sigma \upharpoonright V$ is the
substitution $\mu$ such that if $x$ is in $V$, $x\mu = x\sigma$
and otherwise $x\mu = x$.

From now on we discuss a fixed equational theory $\mathrm{T}$ with
language $L$. In order to allow for Skolem function symbols we make:
\begin{quote}
\emph{Assumption 1}. $L$ contains infinitely many function symbols of
each degree, none of which occur in $\mathrm{T}$.
\end{quote}

Suppose $L$ has the form $(\pm)$$P(t_{1}, \ldots,\,t_{n})$ and
$M$ has the form $(\pm)$$Q(u_{1}, \ldots,\,u_{m})$. Then if $P$
is not the equality symbol, $\vdash_{\mathrm{T}}L \equiv M$ iff
$P$ and $Q$ are identical, $L$ and $M$ have the same sign and,
$\vdash_{\mathrm{T}}t_{i} = u_{i}$,
($1 \leqslant i \leqslant n$).

Simplification and normal-form functions have been defined in the
introduction. They can be applied in the natural way to literals,
clauses, and sets of sets of $\ldots$ such.

\subsection*{Ground level}\label{ground-level}

We will formulate a set of rules, complete at ground level.

Define an equivalence relation, $\sim$, between literals by:
\begin{quote}
If the equality symbol does not occur in $L$, then $L \sim M$ iff
$\vdash_{\mathrm{T}}L \equiv M$.\\
If it occurs with the same sign in both, $L$ is $(\pm)$$t = u$ and
$M$ is $(\pm)$$v = w$, then $L \sim M$ iff
$\vdash_{\mathrm{T}}(t = v)\wedge(u = w)$ or
$\vdash_{\mathrm{T}}(t = w)\wedge(u = v)$.\\
Otherwise, $L \nsim M$.
\end{quote}

A set of literals is a $\mathrm{T}$-\emph{unit} iff $L \sim M$ for any $L$,
$M$ in the set.

If $C''\cup\overline{D}''$ is a $\mathrm{T}$-unit
containing no occurrence of the equality symbol, then
$C'\cup D'$ is a \emph{ground} $\mathrm{T}$-\emph{resolvent} of
$C = C'\lor C''$ and
$D = D'\lor D''$.

If
$C = C'\lor t_{1} = u_{1}\vee\ldots\lor t_{n} = u_{n}(n > 0)$,
$D=D'\lor D''$,
$(\pm)\,P(v_{1},\ldots,v_{j_{0}},$$\ldots,v_{m})$ is in
$D''$, $\{t_{i} = u_{i}\mid i=1,\,n\}$ and
$D''$ are $\mathrm{T}$-units and
$\vdash_{\tau}v_{j_{0}} = w(t_{1})$ then
\begin{quote}
$C'\cup D'\cup\{(\pm)\,P(v_{0},\ldots,v_{j_{0}-1},w(u_{1}/t_{1}),\,v_{j_{0}+1},\ldots,\,v_{m})\}$
\end{quote}
is a \emph{ground} $\mathrm{T}$-\emph{paramodulant} of $C$ and $D$.

If
$C=C'\lor t_{1}\neq u_{1}\lor\ldots\lor\ t_{n}\neq u_{n}(n>0)$
and $\{t_{i}\neq u_{i}\mid i=1,n\}$ is a $\mathrm{T}$-unit and
$\vdash_{\rm T}\,t_{1}=u_{1}$, then $C'$ is a \emph{ground}
$\mathrm{T}$-\emph{trivialisation} of $C$.

These rules are, in general, semi-effective; they are clearly
consistent. If, in an application of one of the above,
$\#\,(C'')=\#\,(D'')=n=1$ ($\#$ is the
cardinality function) the application has \emph{degree} 1.

Given a set, Eq, of ground equations, write $t\approx_{1}u$ iff there
is a term $w(v)$ and another $v'$ such that
$\vdash_{\rm T}\,t=w(v)$, $\{v=v'\}$ is in Eq and
$u=w(v'/v)$. Note that if $t\approx_{1}u$ and
$\vdash_{\rm T}\,u=v$ then for some $w$, $v\approx_{1}w$ and
$\vdash_{\rm T}\,w=t$, that if $\vdash_{\rm T}\,t=u$ and
$u_{1}\approx_{1}w$ then $t\approx_{1}w$, and if
$t_{j}\approx_{1}u$ then
$f(t_{1},\ldots,t_{n})\approx_{1}f(t_{1},\ldots,$$t_{j-1},u$,
$t_{j+1},\ldots,$$t_{n})$.

Now define $\approx$ by: $t\approx u$ iff there are
$t_{i}(n\geqslant 1$ and $1\leqslant i\leqslant n)$ such that

$t=t_{1}\approx_{1}\cdots\approx_{1}t_{n}$ and $\vdash t_{n}=u$.

% Why noindent here?
\noindent
\emph{Lemma 1}. $t \approx u$ iff
$\vdash_{T\cup\mbox{\scriptsize Eq}}t = u$, if $t$ and $u$ are
ground terms.

% Why noindent here?
\noindent
\emph{Proof}. Evidently $t \approx u$ implies
$\vdash_{\mathrm{T} \cup \mathrm{Eq}}t = u$. Conversely, the
properties of $\approx_{1}$ ensure that $\approx$ is a congruence
relation. With the usual square bracket notation for congruence classes,
define an interpretation, $\cal A$, whose domain is the set of
congruence classes and whose operations are given unambiguously by:
\begin{quote}
${\cal A}(f)([t_{1}],\ldots,[t_{n}]) = [f(t_{1},\ldots,\,t_{n})].$
\end{quote}

Every equation in $\mathrm{T}\cup \mathrm{Eq}$ is true under this
interpretation, so as $\vdash_{\mathrm{T} \cup \mathrm{Eq}} t = u$,
then $[t] = {\cal A}(t) = {\cal A}(u) = [u]$; that is,
$t \approx u$.

\noindent
\emph{Theorem 1}. Suppose $S$ is a non-empty set of non-empty ground
clauses such that $S\cup T$ has no model. Then there is a derivation
of the null clause from $S$ using the rules given above, in which all
applications of the rules have degree 1.

\noindent
\emph{Proof}. The proof is by induction on the excess literal parameter
\begin{quote}
$\displaystyle l(S)=\sum_{C\in S}(\#\,(C)-1) \geqslant 0.$
\end{quote}

When $l(S)=0$, $S$ is a set of unit clauses. Let
$\mathrm{Eq}\subseteq S$ be the set of equations in $S$. If every set of
the form $\mathrm{Eq}\cup\{t\neq u\}\cup\mathrm{T}$ ($\{t\neq u\}$
in $S$) or the form
$\mathrm{Eq}\cup\{\{L\}\{M\}\}\cup\mathrm{T} (\{L\}\, , \{M\}$ in
$S)$ is satisfiable, then it can be shown that $S\cup\mathrm{T}$ is
satisfiable. Consequently a set with one such form is unsatisfiable.

In the first case, $\vdash_{\mathrm{Eq}\cup\mathrm{T}} t=u$ and so,
with $\approx$ as in lemma 1, $t\approx u$. So, by successive ground
$\mathrm{T}$-paramodulations from $\textrm{Eq}$ into $t\neq u$,
one can obtain a clause $t'\neq u$ such that
$\vdash_{\mathrm{T}}$$t'=u$. From this the null clause is
obtained by a ground $\mathrm{T}$-trivialisation of degree 1.

The second case is similar but uses a ground $\mathrm{T}$-resolution.

If $l(S)>0$, there is a literal $L$ in a non-unit clause $C$
in $S$. Applying the induction hypothesis to
$S'{=}(S\backslash\{C\})\cup(C\backslash\{L\})$, we obtain a
derivation of the null clause, by the above rules, from $S'$.
So either there is a derivation of the null clause from $S$ by the
rules or there is one of $\{L\}$ by them, from $S$. In the second
case, we obtain, by induction, a refutation of
$S''{=}(S\cup\{L\})\backslash\{C\}$ and the proof is
concluded.

Notice that in the above if $S$ contains no occurrences of the
equality symbol, then a refutation of $S$ can use only ground
$\mathrm{T}$-resolution; similar results hold for other grammatical
possibilities.

A clause $C$, is a \emph{ground N}-$\mathrm{T}$-\emph{resolvent of
	$D$ and $E$} iff $C{=}N(C')$ for some ground
$\mathrm{T}$-resolvent of $D$ and $E$. \emph{Ground
	N}-$\mathrm{T}$-\emph{paramodulation} and \emph{trivialisation} are
similarly defined.

A partial order $\stackrel{{\subset}}{{\sim}}$ is defined between clauses
by:
\begin{quote}
$C\stackrel{{\subset}}{{\sim}}D$ iff there is a function
$\phi \colon C{\rightarrow}\,D$ such that $L{\sim}\,\phi(L)$ for
every $L$ in $C$.
\end{quote}
Note that $N(C)\stackrel{{\subset}}{{\sim}}C$, for any clause $C$

One can check that if $C\stackrel{{\subset}}{{\sim}}C'$,
$D\stackrel{{\subset}}{{\sim}}D'$ and $E'$ is a ground
$\mathrm{T}$-resolvent of $C'$ and $D'$ then
either $C\stackrel{{\subset}}{{\sim}}E'$,
$D\stackrel{{\subset}}{{\sim}}E'$, or
$E\stackrel{{\subset}}{{\sim}}E'$ for some ground
$\mathrm{T}$-resolvent $E$ of $C$ and $D$; similar results hold
for the other two rules.

From these remarks, it follows that if $S$ has a refutation by the T
rules, it has one by the $N$-$\mathrm{T}$ ones; further remarks
analogous to those made immediately after theorem 1 also apply.
(Actually only degree 1 applications of the rules are necessary).

Greater freedom in the application of $N$ is permissible. One can
allow $N$ to depend on the derivation of the clause to which it is
being applied; perhaps $N$ could even depend on the state of execution
of the algorithm searching for a refutation.

\subsection*{Unification}\label{unification}

A substitution, $\sigma$, is a $\mathrm{T}$-\emph{unifier} of
$\cal E$ iff when $t$, $u$ are terms in the same set in
$\cal E$, $\vdash_{\rm T}t\sigma=u\sigma$ and if $L$, $M$ are
literals in the same set in $\cal E$, $L\sigma\sim M\sigma$.

An equivalence relation between substitutions is defined by:
\begin{quote}
$\sigma\sim\mu$ ($\it V$) iff, for all variables, $x$, in
$\it V\vdash_{\rm T}x\sigma=x\mu$.
\end{quote}
Note that, if $\sigma\sim\mu$($\it V$) then
$\sigma v\sim\mu$($\it V$) and if $\sigma\sim\mu(\operatorname{Var}(V_v))$ then
$\it v\sigma\sim v\mu(V)$ for any $\it v$.

A set, $\Gamma$, of substitutions is a \emph{maximally general set of
	$\mathrm{T}$-unifiers} (\mgsu) of $\cal E$ \emph{away from $\it V$}
iff:

\begin{enumerate}
	\def\labelenumi{(\arabic{enumi})}
	\item
	(Correctness). Every $\sigma$ in $\Gamma$ $\mathrm{T}$-unifies
	$\cal E$.
	\item
	(Completeness). If $\sigma'$ $\mathrm{T}$-unifies
	$\cal E$, then there is a $\sigma$ in $\Gamma$ and a $\lambda$
	such that
	$\sigma'\sim\sigma\lambda(\operatorname{Var}(\cal E))$.
	\item
	(Independence). If $\sigma$ and $\sigma'$ are distinct
	members of $\Gamma$, then for no $\lambda$ does
	$\sigma'\sim\sigma\lambda(\operatorname{Var}(\cal E))$.
	\item
	For every $\sigma$ in $\Gamma$ and $x$ not in
	$\operatorname{Var}(\cal E)$, $x\sigma=x$.
	\item
	For every $\sigma$ in $\Gamma$,
	$\operatorname{Var}(\cal E \sigma)\cap V = \varnothing$.
\end{enumerate}

Conditions 4 and 5 are technical: from every set satisfying the first
three conditions one can effectively and, indeed, efficiently construct
one satisfying them all. Note that conditions 2 and 3 use relative
equivalence rather than equality. These conditions can be satisfied in
cases where the corresponding ones with equality cannot, and we know of
no example of the opposite situation.

We also know of no example of a theory $\mathrm{T}$ and an
$\mathcal{E}$ and $V$ for which there is no such $\Gamma$,
although we expect that one exists.

However we make:\\
\emph{Assumption 2}.: There is a partial recursive predicate
$P(\mathcal{E},\sigma,V)$ such that
$\Gamma(\mathcal{E},V)=\{\sigma\mid P(\mathcal{E},\sigma,V)\}$ is a
\mgsu for $\mathcal{E}$ apart from $V$.

On this basis, semi-effective proof procedures can be developed. Of
course in each case we should look for efficient algorithms for
generating $\Gamma$.

We conjecture that if $L_{1}$ and $L_{2}$ are any two languages, not
necessarily satisfying assumption 1 and $L_{1}$ is the language of
$T$ then if there is a (partial recursive, recursive, effectively
obtainable, efficiently obtainable) \mgsu{} for any $\mathcal{E}$, whose
vocabulary is in $L_{1}$ and any $V$ then there is a (partial
recursive, recursive, effectively obtainable, efficiently obtainable)
\mgsu{} for any whose vocabulary is in $L_{1}\cup L_{2}$. This would
eliminate the necessity for assumption 1 and would as a special case,
yield unification algorithms for evaluation systems like the
arithmetical one in the introduction. Indeed if we consider theories
$T_{1}$, $T_{2}$ whose languages are $L_{1}$, $L_{2}$
respectively, such that $L_{1}\cap L_{2} = \varnothing$, it may be the
case that \mgsu's for $T_{1}\cup T_{2}$ always exist
(and are partial recursive, etc.) if the same holds for $T_{1}$ and
$T_{2}$ individually.

The algorithms given in the introduction do not quite conform to the
specifications in that they do not give unifiers for an arbitrary
$\mathcal{E}$ and, further, violate condition 5, in general. However,
this is easily corrected. Interestingly, instead of condition 5, they
satisfy:
\begin{quote}
(5') For every $\sigma$ in $\Gamma$,
$\operatorname{Var}\,(\mathcal{E}\sigma)\subseteq\operatorname{Var}\,(\mathcal{E})$.
\end{quote}

This condition allows the subsequent theory to proceed with some
modifications, but can\textquotesingle t be fulfilled for some
$\mathrm{T}$\textquotesingle s for which 5 can.

\mgsu's are essentially unique. If $\Gamma$,
$\Gamma'$ are \mgsu's for $\mathcal{E}$ away
from $V$ and $V'$, say, there is a bijection
$\phi\colon\Gamma{\rightarrow}\Gamma'$ such that for all
$\sigma$ in $\Gamma$ there are $\lambda$, $\lambda'$
such that
$\sigma{\sim}\phi(\sigma)\lambda(\operatorname{Var}(\mathcal{E}))$ and
$\phi(\sigma){\sim}\sigma\lambda' (\operatorname{Var}(\mathcal{E}))$.

Again, if $\mathcal{E}$ and $\mathcal{E}'$ have the same set
of $\mathrm{T}$-unifiers and variables, $\Gamma$ is a \mgsu{} for $\mathcal{E}$ away from $V$ iff it is for $\mathcal{E}'$. This holds, in
particular, if $\mathcal{E}' = N(\mathcal{E})$.

A change of variables in $\mathcal{E}$ causes a corresponding one in
$\Gamma$. If $\Gamma$ is a \mgsu{} for $\mathcal{E}$ away from $V$
then if $\xi$ maps distinct variables of $\mathcal{E}$ to distinct
variables,
$\{(\xi^{-1}\sigma) \upharpoonright \operatorname{Var}(\mathcal{E}\xi)\mid \sigma\in\Gamma\}$
is a \mgsu{} for $\mathcal{E}\xi$ away from $V$.

Before defining suitable generalisations of resolution and so on, one
should decide between formulations with or without factoring. Usually
the difference is one, merely, of efficiency; the justification lies in
a theorem of Hart (1965) (see also Kowalski (1970)), that if $\sigma$
is a m.g.u. of $\mathcal{E}$ and $\sigma'$ is of
$\mathcal{E}'\sigma$ then $\sigma\sigma'$ is of
$\mathcal{E}\cup\mathcal{E}'$. Unfortunately the
generalisation of this theorem fails. We can prove:

\noindent
\emph{Theorem 2} (\emph{Weak Refinement Theorem}). Suppose $\Gamma$
is a \mgsu{} for $\mathcal{E}_{1}$ apart from
$\operatorname{Var}\left(\mathcal{E}_{2}\right)\cup V$ and for each
$\sigma$ in $\Gamma$, $\Gamma_{\sigma}$ is a \mgsu{} for
$\mathcal{E}_{2}\sigma$, apart from
$\operatorname{Var}(\mathcal{E}_{1}\sigma)\cup V$.

Then,
$\Gamma' = \{\sigma\mu \upharpoonright \operatorname{Var}(\mathcal{E}_{1}\cup \mathcal{E}_{2})\mid \sigma \in \Gamma,\, \mu \in \Gamma_{\sigma}\}$
satisfies conditions 1, 2, 4 and 5 for being an \mgsu{} for
$\mathcal{E}_{1}\cup\mathcal{E}_{2}$ away from $V$.

\noindent
\emph{Proof}. Conditions 1, 4 and 5 may be straightforwardly verified.
To establish condition 2, suppose $\theta$ $\mathrm{T}$-unifies
$\mathcal{E}_{1}\cup\mathcal{E}_{2}$. Then by condition 2 on
$\Gamma$, there is a $\sigma$ in $\Gamma$ and a $\lambda$ such
that:
\begin{quote}
$\theta \sim \sigma\lambda(\operatorname{Var}(\mathcal{E}_{1}))$.
\end{quote}

It can be assumed, without loss of generality, that
$\lambda = \lambda \upharpoonright \operatorname{Var}(\mathcal{E}_{1}\sigma)$. Now
$\operatorname{Var}(\mathcal{E}_{2})\cap\operatorname{Var}(\mathcal{E}_{1} \sigma) = \varnothing$
by hypothesis and condition 5 on $\Gamma$. So
$\lambda' = \lambda\cup(\theta \upharpoonright \operatorname{Var}(\mathcal{E}_{2}))$
is a substitution. Now:
\begin{quote}
$\theta \sim \sigma\lambda' (\operatorname{Var}(\mathcal{E}_{1}\cup\mathcal{E}_{2}))$.
\end{quote}

For if $x$ is in $\operatorname{Var}(\mathcal{E}_{1})$ then
$\operatorname{Var}(x\sigma)\cap\operatorname{Var}(\mathcal{E}_{2}) = \varnothing$,
so $x\sigma\lambda' = x\sigma\lambda$ and
$\vdash_{\operatorname{T}}\,x\sigma\lambda = x\theta$. If $x$ is in
$\operatorname{Var}(\mathcal{E}_{2})$ but not in
$\operatorname{Var}(\mathcal{E}_{1})$ then $x\sigma = x$, by
condition 4 on $\Gamma$ and $x\lambda = x$ by assumption. So
$x\lambda' = x(\theta \upharpoonright \operatorname{Var}(\mathcal{E}_{2})) = x\theta$.

Therefore $\sigma\lambda$ $\mathrm{T}$-unifies
$\mathcal{E}_{1}\cup\mathcal{E}_{2}$, $\lambda$
$\mathrm{T}$-unifies $\mathcal{E}_{2}\sigma$ and so by condition 2
on $\Gamma_{\sigma}$, there is a $\mu$ in $\Gamma_{\sigma}$ and a
$\delta$ such that:
\begin{quote}
$\lambda' \sim \mu\delta (\operatorname{Var}(\mathcal{E}_{2}\sigma))$
\end{quote}

It can be assumed without loss of generality that
$\delta\upharpoonright\mathrm{Var}(\mathcal{E}_{2}\sigma\mu)=\delta$.
Now,
$\mathrm{Var}(\mathcal{E}_{1}\sigma)\cap\mathrm{Var}(\mathcal{E}_{2}\sigma\mu) =\varnothing$,
by hypothesis and condition 5 on $\Gamma_{\sigma}$. So
$\delta'=\delta\cup(\lambda'\upharpoonright\mathrm{Var}( \mathcal{E}_{1}\sigma))$
is a substitution and one can show much as above that:
\begin{quote}
$\theta\sim\sigma\lambda'$ ($\operatorname{Var}(\mathcal{E}_{1}\cup\mathcal{E}_{2})$).
\end{quote}


For if $x$ is in $\operatorname{Var}(\mathcal{E}_{1})$ then $\operatorname{Var}(x\sigma)\cap\operatorname{Var}(\mathcal{E}_{2})=\emptyset$, so $x\sigma\lambda'=x\sigma\lambda$ and $\Gamma_{\operatorname{T}}\,x\sigma\lambda=x\theta$. If $x$ is in $\operatorname{Var}(\mathcal{E}_{2})$ but not in $\operatorname{Var}(\mathcal{E}_{1})$ then $x\sigma=x$, by condition 4 on $\Gamma$ and $x\lambda=x$ by assumption. So $x\lambda'=x(\theta \upharpoonright \operatorname{Var}(\mathcal{E}_{2}))=x\theta$.

Therefore $\sigma\lambda$ $\mathrm{T}$-unifies $\mathcal{E}_{1}\cup\mathcal{E}_{2}$, $\lambda$ $\mathrm{T}$-unifies $\mathcal{E}_{2}\sigma$ and so by condition 2 on $\Gamma_{\sigma}$, there is a $\mu$ in $\Gamma_{\sigma}$ and a $\delta$ such that:
\begin{quote}
$\lambda'\sim\mu\delta(\operatorname{Var}(\mathcal{E}_{2}\sigma))$
\end{quote}

It can be assumed without loss of generality that $\delta\upharpoonright\mathrm{Var}(\mathcal{E}_{2}\sigma\mu)=\delta$. Now, $\mathrm{Var}(\mathcal{E}_{1}\sigma)\cap\mathrm{Var}(\mathcal{E}_{2}\sigma\mu) =\varnothing$, by hypothesis and condition 5 on $\Gamma_{\sigma}$. So $\delta'=\delta\cup(\lambda'\upharpoonright\mathrm{Var}( \mathcal{E}_{1}\sigma))$ is a substitution and one can show much as above that:
\begin{quote}
$\lambda'\sim\mu\delta'$ \quad$(\mathrm{Var}((\mathcal{E}_{1}\cup\mathcal{E}_{2})\sigma))$
\end{quote}
Then $\sigma\lambda'\sim\sigma\mu\delta$ \quad $(\mathrm{Var}(\mathcal{E}_{1}\cup\mathcal{E}_{2}))$ and:

\begin{quote}
$\theta \sim \sigma\lambda' \sim \sigma(\mu \delta') \sim (\sigma\mu\upharpoonright\mathrm{Var}(\mathcal{E}_{ 1}\cup\mathcal{E}_{2}))\delta'$   \quad$(\mathrm{Var}(\mathcal{E}_{1}\cup \mathcal{E}_{2}))$,
\end{quote}
concluding the proof.

When T is the theory of an associative, commutative idempotent function,
one can find examples where condition 3 fails. We will therefore
formulate the rules without using factoring. (Factoring would still
result in completeness, but would cause redundancy which although
eliminable by a subsumption-type check would probably cause more trouble
than it was worth.) Condition 3 does hold, however, in the associative
and evaluation examples given in the introduction.

Gould (1966) has defined, in the context of $\omega$-order logic, the
concept of a general matching set.

In our terms, a literal, $M$, is a $\mathrm{T}$-\emph{unification} of $L$ and
$L'$ iff for some $\sigma$,
$L\sigma\sim M\sim L'\sigma$.

Then, $\Delta$ is a \emph{general matching set of literals} for $L$
and $L'$ away from $V$ iff:

\begin{enumerate}
	\def\labelenumi{(\arabic{enumi})}
	\item
	Every member of $\Delta$ is a $\mathrm{T}$-unification of $L$
	and $L'$.
	\item
	If $M'$ is a $\mathrm{T}$-unification of $L$ and
	$L'$, then there is a $\lambda$ and an $M$ in $\Delta$
	such that $M\lambda\sim M'$.
	\item
	If $M$, $M'$ are distinct members of $\Delta$ then for
	every $\lambda$, $M\lambda\nsim M'$.
	\item
	If $M$ is in $\Delta$, Var $(M)\cap$$V$=$\varnothing$.
\end{enumerate}

In our opinion, this concept cannot, in general, be made the basis of a
complete inference system; a counterexample will be given later.

\subsection*{General Level}\label{general-level}

We begin with the rules.

Suppose $\sigma$ is in
$\Gamma(\{C''\xi_{C,D}\cup\overline{D}''\}, \operatorname{Var}
(C\xi_{C,D}\cup D))$ where $C=C'\vee C''$,
$D=D'\lor D''$ and the equality symbol has no
occurrence in $C''$. Then,
\begin{quote}
$C'\xi_{C,D}\sigma\,\cup\,D'\sigma$ is a
$\mathrm{T}$-\emph{resolvent} of $C$ and $D$.
\end{quote}

A variable-term pair is \emph{special} iff it is of the form
$\langle x,x\rangle$ or else
$\langle x, f(x_{1},$ $\ldots,$ $x_{i-1},$ $t,$ $x_{i+1},$ $\ldots,$ $x_{n})\rangle$ $(n>0)$
and $\langle x, t\rangle$ is special and the $x_{i}$ are distinct
and not in $t$.

Given an occurrence of a term $u$ in another $v$ there is a unique,
to within alphabetic variance, special pair $\langle x,$$t\rangle$
and a substitution $\eta$ such that:

\begin{enumerate}
	\def\labelenumi{(\arabic{enumi})}
	\item
	$v=t\{u/x\}\eta,$
	\item
	$\{u/x\}\eta=\eta\{u/x\},$
	\item
	$x$ has the same occurrence in $t$ as $u$ in $v,$ and
	\item
	$\mbox{Var}(t)\cap\mbox{Var}(v)=\varnothing$.
\end{enumerate}

We assume available a function, $Sp,$ from finite sets of variables to
special pairs such that:

\begin{enumerate}
	\def\labelenumi{(\arabic{enumi})}
	\item
	There is an alphabetic variant of every special pair in $Sp(V).$
	\item
	No two distinct members of $Sp(V)$ are alphabetic variants.
	\item
	If $\langle x,$$t\rangle$ is in $Sp(V),$ Var
	$(t)\cap V=\varnothing$.
\end{enumerate}

Suppose
$C=C'\lor t_{1}=u_{n}\vee\ldots\lor\ t_{n}=u_{n}(n>0)$,
$D=D'\lor D''$,
$(\pm)P(v_{1}, \ldots, v_{j_{0}}, \ldots, v_{m})$ is in
$D''$, $\langle x, t\rangle$ is in
$Sp(\operatorname{Var}(C\xi_{C,D}\cup D))$ and that
${\cal E}=\{\{t_{i}\xi_{C,D}\mid i=1,\,n\}$,
$\{u_{i}\xi_{C,D}\mid i=1, n\}$, $D''$,
$\{v_{j_{0}}, t\{t_{1}\xi_{C, D}\mid x\}\}\}$.

Then, if $\sigma$ is in
$\Gamma({\cal E}, \operatorname{Var}\,\,(C\xi_{C,D}\cup D))$,

$C'\xi_{c,D}\sigma\cup D'\sigma\cup\{(\pm)P(v_{1}\sigma,\ldots,v_{j_{0}-1}\sigma,t\{u_{1}\xi_{C,D} / x\}\sigma,v_{j_{0}+1}\sigma,\ldots,v_{m}\sigma)$\\
is an \emph{assisted primary} $\mathrm{T}$-\emph{paramodulant from $C$ into $D$.}

Suppose that
$C=C'\lor t_{1}\neq u_{1}\ldots\lor\ t_{n}\neq u_{n}(n>0)$ and
$\sigma$ is in
$\Gamma(\{\{t_{i}\mid i=1,n\}\cup\{v_{i}\mid i=1,\,n\}\},\operatorname{Var}(C))$;
then, $C'\sigma$ is a $\mathrm{T}$-\emph{trivialisation} of $C$.

These three rules are evidently consistent and semi-effective.

For completeness a lifting lemma is needed.

\noindent
\emph{Lemma 2}

\begin{enumerate}
	\def\labelenumi{(\arabic{enumi})}
	\item
	Suppose $E'$ is a ground $\mathrm{T}$-resolvent of ground clauses
	$C\mu$ and $Dv$. Then there is a $\mathrm{T}$-resolvent, $E$ of $C$ and
	$D$ and a $\lambda$ such that $E\lambda \stackrel{{\subset}}{{\sim}} E'$.
	\item
	Suppose $E'$ is a ground $\mathrm{T}$-paramodulant from the ground
	clause $C\mu$ into the ground clause $Dv$. Then there is an
	assisted primary $\mathrm{T}$-paramodulant, $E$, from $C$ into $D$ and a
	$\lambda$ such that $E\lambda \stackrel{{\subset}}{{\sim}}a E'$.
	\item
	Suppose $E'$ is a ground $\mathrm{T}$-trivialisation of the ground
	clause $C\mu$. Then there is a $\mathrm{T}$-trivialisation, $E$, of $C$ and
	a $\lambda$ such that $E\lambda \stackrel{{\subset}}{{\sim}} E'$.
\end{enumerate}

\noindent
\emph{Proof.} Only the proof of part 2, the hardest one, will be given.

Let
$\sigma'=(\xi_{C,D}^{-1}\mu\upharpoonright Var(C\xi_{C,D}))\cup(v \upharpoonright Var(D))$.

Then $C\xi_{C,D}\sigma'=C\mu$ and $D\sigma'=Dv$. So,
as $E'$ is a ground $\mathrm{T}$-paramodulant of
$(C\xi_{C,D})\sigma'$ and $D\sigma'$ there are
subsets $C'$, $D'$ and $D''$ of $C$ and
$D$, respectively, and terms $t_{i}$, $u_{i}$ ($i=1$, $n$),
$t$, $u$, $v'_{j0}$, $v'(t)$ and a literal
$(\pm)\,P(v'_{1},\ldots,v'_{j0},\ldots,v'_{m})$
in $D''\sigma'$ such that:

$$
\begin{aligned}
	& C \xi_{C, D} \sigma'=C' \xi_{C, D} \sigma' \vee\left\{\left(t_i=u_i\right) \xi_{C, D} \sigma' \mid i=1, n\right\}, \\
	& D \sigma'=D' \sigma' \vee D'' \sigma', \\
	& t_1 \xi_{C, D} \sigma'=t \\
	& u_1 \xi_{C, D} \sigma'=u \\
	& \vdash_T v_{j_0}'=v'(t), \\
	& E'=C' \xi_{C, D} \sigma' \vee D' \sigma' \vee( \pm) P\left(v_1', \ldots, v_{j_0 - 1}', v'(u / t), v_{j_0+1}', \ldots, v_m'\right) \\
	& \left\{\left(t_i=u_i\right) \xi_{C, D} \sigma' \mid i=1, n\right\} \text { and } D'' \sigma' \text { are } \mathrm{T}\text{-units, } \\
	& C=C' \vee t_1=u_1 \vee \ldots \vee t_n=u_n, \\
	& D=D' \vee D'', \\
	& \vdash_T t_i \xi_{C, D} \sigma'=t \quad(i=1, n) \text { and } \\
	& \vdash_T u_i \xi_{C, D} \sigma'=u \quad(i=1, n) .
\end{aligned}
$$

As $t$ is a subterm of $v'$, there is a unique $\langle x, w\rangle$ in ${Sp}\left(\operatorname{Var}\left(\left(C \xi_{C, D} \cup D\right)\right)\right)$ and an $\eta$ such that:
$$
v'=w\{t / x\} \eta,\{t / x\} \eta=\eta\{t / x\} \text { and } x \text { has the same occurrence in } w
$$
as the distinguished occurrence of $t$ in $v'$.

All the above equations hold if $\sigma'$ is replaced by $\sigma''=\sigma' \eta$, as $C \mu$ and $D v$ are ground.

Now, let $( \pm) P\left(v_1, \ldots, v_m\right)$ be a literal in $D''$ such that $( \pm) P\left(v_1, \ldots, v_m\right) \sigma'$ $=( \pm) P\left(v_1', \ldots, v_m'\right)$. We have,
$$
\begin{aligned}
	v_{j_0} \sigma'' & =v_{j_0}', \\
	\vdash_{\mathrm{T}} v_{j_0}' & =v'(t) \text { and } \\
	v'(t) & =w\{t / x\} \eta=w\left\{t_1 \xi_{C, D} \sigma' / x\right\} \eta \\
	& =w\left\{t_1 \xi_{C, D} / x\right\} \sigma' \eta \quad\left(\operatorname{as} \sigma' \uparrow \operatorname{Var}(w)=\varepsilon\right) \\
	& =w\left\{t_1 \xi_{C, D} / x\right\} \sigma'' .
\end{aligned}
$$

Therefore, $\vdash_{\mathrm{T}} v_{j_0} \sigma''=w\left\{t_1 \xi_{C, D} / x\right\} \sigma''$, and we have proved that $\sigma''$ $\mathrm{T}$-unifies $\mathcal{E}=\left\{\left\{t_i \xi_{C, D} \mid i=1, n\right\},\left\{u_i \xi_{C, D} \mid i=1, n\right\}, D'',\left\{v_{j_0}, w\left\{t_1 \xi_{C, D} / x\right\}\right\}\right\}$. So there is a $\sigma$ in $\Gamma\left(\mathcal{E}, \operatorname{Var}\left(\left(C \xi_{C, D} \cup D\right)\right)\right)$ and a $\lambda'$ such that:
$$
\sigma'' \sim \sigma \lambda'(\operatorname{Var}(\mathcal{E})) \text {. }
$$

One then finds, as in the proof of theorem 2 , a $\lambda$ such that
$$
\sigma'' \sim \sigma \lambda\left(\operatorname{Var}\left(\left(C \xi_{C, D} \cup D\right)\right)\right. \text {. }
$$

Now $E=C' \xi_{C, D} \sigma \cup D' \sigma \cup\left\{( \pm) P\left(v_1 \sigma, \ldots, v_{j_0-1} \sigma, w\left\{u_1 \xi_{C, D} / x\right\} \sigma, v_{j_0+1} \sigma\right.\right.$, $\left.\left.\ldots, v_m \sigma\right)\right\}$, is an assisted primary $\mathrm{T}$-paramodulant from $C$ into $D$.

$$
\begin{aligned}
	\text{Since } w\left\{u_1 \xi_{C, D} / x\right\} \sigma'' & =w\left\{u_1 \xi_{C, D} / x\right\} \sigma' \eta \\
	& =w\left\{u_1 \xi_{C, D} \sigma' / x\right\} \eta \quad\left(\text{as } \sigma' \upharpoonright \operatorname{Var}(w)=\varepsilon\right) \\
	& =w\{u / x\} \eta \\
	& =v'(u / t),
\end{aligned}
$$
$$
\begin{aligned}
	E \lambda &=C' \xi_{C, D} \sigma \lambda \cup D' \sigma \lambda \cup\left\{( \pm ) P \left(v_1 \sigma \lambda, \ldots, v_{j_0-1} \sigma \lambda, w\left\{u_1 \xi_{C, D} / x\right\} \sigma \lambda,\right.\right. \\
	& \qquad \qquad \left.\left.v_{j_0+1} \sigma \lambda, \ldots, v_m \sigma \lambda\right)\right\} \\
	& \subseteq C' \xi_{C, D} \sigma'' \cup D' \sigma'' \cup\left\{( \pm ) P \left(v_1 \sigma'', \ldots, v_{j_0-1} \sigma'', w\left\{u_1 \xi_{C, D} / x\right\} \sigma''\right.\right. \\
	& \qquad  \qquad \left.\left.v_{j_0+1} \sigma'', \ldots, v_m \sigma''\right)\right\} \\
	& =C' \xi_{C, D} \sigma'' \cup D' \sigma'' \cup\left\{( \pm) P\left(v_1', \ldots, v_{j_0-1}', v'(u / t), v_{j_0+1}', \ldots, v_m'\right)\right\} \\
	& =E' \text {, concluding the proof. } \\
	&
\end{aligned}
$$

\noindent
\emph{Theorem 3}. If $S$ is a
non-empty set of non-empty clauses such that $S\cup T$ has no model,
there is a derivation of the null clause from $S$ using the rules
given above.

\noindent
\emph{Proof}. First it is shown by induction on derivation size (that
is, the number of applications of rules) that if there is a derivation
of a clause $E'$ from a set of ground clauses of the form
$\bigcup\limits_{i}(S\sigma_{i})$, using the ground rules, then there
is a general derivation of a clause $E$ from $S$ and a $\lambda$
such that $E\lambda\stackrel{{\subset}}{{\sim}}E'$.

When the derivation has size zero this is immediate.

Otherwise either there is a derivation of a ground clause $C'$
from a set of clauses of the form $\bigcup\limits_{i}(S\sigma_{i})$
and another of a ground clause $D'$ from a set of the form
$\bigcup\limits_{j}(S\sigma_{j}')$, each derivation having
strictly smaller size than the main derivation under consideration and
$E'$ is a ground $\mathrm{T}$-resolvent of $C'$ and
$D'$, or a corresponding statement involving one of the other
two ground rules is true.

Let us suppose the first case holds. Then, by induction there are
general derivations of clauses $C$ and $D$ from $S$ and,
substitutions $\mu$ and $v$ such that
$C\mu\stackrel{{\subset}}{{\sim}}C'$ and
$Dv\stackrel{{\subset}}{{\sim}}D'$. By a remark made
after theorem 1 either
$C\mu\stackrel{{\subset}}{{\sim}}E'$,
$Dv\stackrel{{\subset}}{{\sim}}E'$ or there is a
ground $\mathrm{T}$-resolvent $E''$ of $C\mu$ and $Dv$ such that
$E''\stackrel{{\subset}}{{\sim}}E'$. In the
first two subcases, we are finished with this case. In the third, by
lemma 2.1 there is a $\mathrm{T}$-resolvent, $E$, of $C$ and $D$ and a
$\lambda$ such that
$E\lambda\stackrel{{\subset}}{{\sim}}E''$, concluding
this case.

A similar proof works in the other two cases.

Now, as $S\cup T$ has no model, there are, by
Herbrand's Theorem, substitutions $\sigma_{i}$, such
that $\bigcup\limits_{i}(S\sigma_{i})$ is ground and
$\bigcup\limits_{i}(S\sigma_{i})\cup T$ has no model. Hence by theorem
1, there is a ground derivation of the null-clause from
$\bigcup\limits_{i}(S\sigma_{i})$. By the above, there is a general
derivation of a clause $E$ from $S$ and a $\lambda$ such that
$E\lambda\stackrel{{\subset}}{{\sim}}\varnothing$. Then
$E$=$\varnothing$, concluding the proof.

Kowalski (1968) proved that, when T = $\varnothing$ one need only
paramodulate into the functional reflexivity axioms, not from them, and
all other paramodulations can be primary. Theorem 3 strengthens this
result: special pairs take the place of the functional reflexivity
axioms. One can also insist on $P1$ and $A$-ordering restrictions,
analogous to Kowalski\textquotesingle s, without losing completeness.

Other refinements are possible. For example one can define an
$E$-T-resolution rule and a corresponding $E$-$\mathrm{T}$-trivialisation one
analogous to $E$-resolution (Morris, 1969). These two rules form a
complete set. If one regards an $E$-T-resolution as a many-input rule,
rather than a deduction using several $\mathrm{T}$-paramodulations and a
T-resolution (and similarly for $E$-$\mathrm{T}$-trivialisation), and says that a
clause has \emph{support} if it is in the support set or is obtained by
a rule application with a clause with support in its input set, the
resulting set of support strategy is complete. (Anderson (1970) showed
that a stronger one is incomplete). Presumably, although we have not
checked, one can obtain systems analogous to the other ones for
equality.

As in the ground case, when there is no occurrence of the equality
symbol in $S$, there is a refutation using only $\mathrm{T}$-resolution, with
similar things holding for the other grammatical possibilities. When
there are no equality symbols occurring in $S$, refinements analogous
to all the usual refinements of the resolution principle hold.

Again, using a simplification function $N$, one can define
$N$-$\mathrm{T}$-resolution and so on and obtain $N$-$\mathrm{T}$-versions of all the
above, by methods like those used in the ground case.

We do not know how to formulate the paramodulation conjecture, as the
obvious way fails.

If $C=C'\lor C''$ and $\sigma$ is in
$\Gamma(\{C\},\,\mbox{Var}\,\,(C))$, then if $L$ is in
$C''$, $C'\sigma\cup\{L\sigma\}$ is a $\mathrm{T}$-\emph{factor}
of $C$, with \emph{distinguished literal} $L\sigma$.

If $C'=C''\lor t=u$ is a $\mathrm{T}$-factor of $C$ with
distinguished literal $t=u$,
$D''=D'\vee(\pm)P(v_{1},\ldots,v_{j_{0}},\ldots,v_{m})$
is a $\mathrm{T}$-factor of $D$ with distinguished literal
$(\pm)P(v_{1},\ldots,v_{m})$, $w(t')$ is a term such that
$\vdash_{\mathrm{T}}v_{j_{0}}=w(t')$ and $\sigma$ is in
$\Gamma(\{\{t,t'\}\})$, Var
$(C'\xi_{C',D'}\cup D'))$ then
\begin{quote}
$C''\xi_{C',D'}\sigma\cup D'\sigma\cup\{(\pm)P(v_ {1},\ldots,w(u\xi_{C,D}/t'),\ldots,v_{m})\sigma\}$
\end{quote}
is a $\mathrm{T}$-\emph{paramodulant from} $C$ \emph{into} $D$.

However, suppose $\mathrm{T}$ is the theory of an associative function, then if
$S=\{f(bx,x)\neq f(xa,caa),bc=ca\}$, $\mathrm{T}$ is unsatisfiable but has no
refutation by $\mathrm{T}$-resolution, $\mathrm{T}$-trivialisation and $\mathrm{T}$-paramodulation.

A clause $C$, is a $\mathrm{T}$-tautology iff $\vdash_{\mathrm{T}}C$. We
don\textquotesingle t know if any deletion strategies work, even when
$\mathrm{T} = \varnothing$. If there are non-equality literals $L$, $M$
in $C$ such that $L \sim M$, or a literal $t = u$ in $C$
such that $\vdash_{\mathrm{T}}t = u$, then $C$ is a \emph{weak
	$\mathrm{T}$-tautology}. The usual deletion strategies altered to take account of
the extra rules work in this case.

A clause $C = \{L_{i}\}$ $\mathrm{T}$-\emph{subsumes} another
$D = \{M_{j}\}$ iff there is a $\sigma$ such that
$\vdash_{\mathrm{T}}(\vee_i L_{i}\sigma) \supset ( \vee_i M_{j})$.
We do not know if any deletion strategies work, even when
$\mathrm{T} = \varnothing$. If $C\sigma \subseteq D$ for some $\sigma$
then $C$ \emph{weakly} $\mathrm{T}$-\emph{subsumes} $D$. Appropriate
alterations of the usual deletion strategies work in this case.

Here is an example showing why we do not consider that the concept of a
general matching set of literals (or whatever) can be made the basis of
a complete system of inference.

Consider the rule:

Let $C'=C''\lor L$ and
$D'=D''\lor L'$ be $\mathrm{T}$-factors of clauses
$C$ and $D$, with distinguished literals $L$ and $L'$
respectively. If $M$ is in a general matching set of literals for
$L\xi_{C', D'}$, and $\bar{L'}$ away from
$\mbox{Var} (C'\xi_{C', D'}\cup D')$,
$\sigma$ is such that
$L\xi_{C', D'}\sigma\sim M\sim\bar{L'}\sigma$,
and the equality symbol does not occur in $L$, then
$C''\xi_{C', D'}\sigma\cup D''\sigma$
can be deduced from $C$ and $D$ by the rule.

But if T is the theory of an associative binary function, recursive
effectively obtainable general matching sets for $L$ and
$L'$ away from $V$ exist, for all $L$, $L'$and
$V$, and furthermore the required $\sigma$\textquotesingle s in the
definition of the rule can be calculated from
$L'\xi_{C', D'}$, $M$ and $L'$.

However, if $S$=$\{\overline{P}\left(ax,aa\right)$,
$\overline{Q}\left(aa,ax\right),P\left(yz,x\right)\lor Q\left(y,xw\right)\}$
then $S$$\cup$$T$ is unsatisfiable but $S$ has no refutation by
the above rule.

\subsection*{Efficiency}\label{efficiency}

The most obvious difficulty is that there can be infinitely many
$\mathrm{T}$-resolvents, etc., of two clauses. This does not cause any difficulty
as far as the Hart-Nilsson \emph{theory} is concerned (Kowalski 1972).
It may be possible in some cases to represent the infinite set by using
terms with parameters. These have already been used informally in the
introduction.

Sometimes, as when T is the theory of an associative function, the
unifiers are generated as successful nodes in a search tree. In this
case the unification search tree can be incorporated in the general
search space and one can save the unused parts.

However, we would like to advance the thesis that in general these large
numbers of unifiers are present in the usual search spaces - we have
just brought them to the surface. Indeed we believe that our method can
have advantages over the standard resolution one precisely analogous to
those obtained by the resolution method over the Davis-Putnam one. We
will defend our method in the case where the theory is that of an
associative function; the $\mathrm{T}$-search space will be compared to that
generated by resolution, para-modulation and trivialisation.

It can be shown that the associative unification of two literals, say,
may be simulated by successive paramodulations of the axiom of
associativity into these literals followed by an ordinary unification.
The complexity of the simulation is about the same as that of the
unification, using a symbol count. So the $\mathrm{T}$-search space is included in
the usual one.

The resolution method has smaller redundancy and avoids more
irrelevancies than the Davis-Putnam method. For example, if $L$ and
$M$ are unifiable then in general $\{\overline{L}, M\}$ has exactly
one resolution refutation, but infinitely many Davis-Putnam ones. The
crudest resolution search procedure produces no irrelevancies but the
Davis-Putnam will, in general. If $L, M$ are not unifiable, this will
be detected by the resolution method, but not by the Davis-Putnam one.
These phenomena are manifestations of the most general unifier method.

Similarly, if $L$ and $M$ have $\mathrm{T}$-unifications,
$\{\overline{L},M\}$ can have many fewer $\mathrm{T}$-resolution refutations than
the comparison system. For example $\{\overline{P}\left(ax\right)$,
$P\left(yb\right)\}$ have two $\mathrm{T}$-unifiers ($\{a/y$, $b/x\}$ and
$\{ay/y,yb/x\}$ but there are infinitely many standard refutations of
$\{\overline{P}\left(ax\right),P\left(yb\right)\}$, which essentially
produce the $\mathrm{T}$-unifiers, $\{ay_{1}\ldots y_{n}/y$,
$y_{1}\ldots y_{m}b/x \mid m$$\geqslant$$0\}$. Each of these
unifiers can be produced in many different ways, involving arbitrarily
long detours of bracket swapping.

As another example, consider
$\{\overline{P}\left(x,y,xy\right),P\left(z,w,wz\right)\}$ which has
aninfinite set of refutations involving the $\mathrm{T}$-unifiers
$\{\{w^{m}/z,w^{m}/x,w^{n}/y,w^{n}/w\} \mid m$ and $n$ positive
integers with greatest common divisor unity$\}$. The standard method
essentially produces the denser set:
\begin{quote}
$\{\{w^{m}/z, w^{m}/x, w^{n}/y, w^{n}/w\}\mid m, n>0\}$.
\end{quote}

Sometimes if $L$ and $M$ have no $\mathrm{T}$-unifier, the
$\mathrm{T}$-unification algorithm will stop, when the standard
procedure does not -- for example, if $L=P(xa)$ and
$M=\overline{P}(yb)$; but generally it will also generate useless
structures (though not so many).

We believe that these informal remarks can be converted into a rigorous
proof of increased efficiency.

On the other hand, the associative unification procedure can certainly
be greatly improved, and we have no practical experience at the moment.
It is surely not the case that these methods will by themselves make a
practical theorem-prover, of course. We have only removed one of many
exponential explosions.

\subsection*{Problems}\label{problems}

There are many obvious problems concerning particular axiom systems and
how to combine different unification procedures.

However, what does one do with only partial knowledge? Suppose one has a
simplification method, but no unification algorithm as is the case, at
the moment, with group theory or with integration algorithms (Moses
1967)? Or, what use can be made of a one-way unification procedure? How
and when does one simplify (Moses 1971)? Answering such questions might
produce more efficient systems closer to normal human practice.

\section*{Acknowledgements}\label{acknowledgements}

I have had many helpful discussions with Rod Burstall. Jerry Schwarz
found some mistakes in a draft. The work was supported by the Science
Research Council.

\section*{References}\label{references}

\begin{itemize}[label={}, itemsep=8pt]
	\item
	Anderson (1970) Anderson, R. (1970) Completeness results for
	E-resolution. \emph{Proc. AFIPS 1970 Spring Joint Comp. Conf.}
	Washington, DC.
	\item
	Bennett et al. (1967) Bennett, J. H., Easton, W. B., Guard, J. R. and
	Settle, L. G. (1967) C.R.T.-aided semi-automated mathematics. Final
	report. aferl-67-0167. Princeton: Applied Logic Corporation.
	\item
	Cook (1965) Cook, S. A. (1965) Algebraic techniques and the
	mechanisation of number theory. r m-4319-pr. California, Santa Monica:
	rand Corporation.
	\item
	Gould (1966) Gould, W. E. (1966) A matching procedure for
	$\omega$-order logic. \emph{Sci. Rep. No. 4. a} fcrl 66-781.
	Princeton, New Jersey: Applied Logic Corporation.
	\item
	Hart (1965) Hart, T. P. (1965) A useful property of
	Robinson\textquotesingle s unification algorithm. \emph{A.I. Memo 91}.
	Project mac. Cambridge, Mass.: mit.
	\item
	Kowalski (1968) Kowalski, R. A. (1968) The case for using equality
	axioms in automatic demonstration. \emph{Lecture Notes in
		Mathematics}, vol. 125. (eds Laudet, M., Nolin, L. and Schutzenberger,
	M.) Berlin: Springer-Verlag.
	\item
	Kowalski (1970) Kowalski, R. (1970) Studies in the completeness and
	efficiency of theorem-proving by resolution. Ph.D. Thesis. Department
	of Computational Logic, University of Edinburgh.
	\item
	Kowalski (1971)Kowalski, R. K. (1972) And-or graphs, theorem-proving
	graphs and bi-directional search. Machine Intelligence 7, paper 10
	(eds Meltzer, B. \& Michie, D.). Edinburgh: Edinburgh University
	Press.
	\item
	Morris (1969) Morris, J. B. (1969) E-resolution: extension of
	resolution to include the equality relation. \emph{Proc. First Int.
		Joint Conf. on Art. Int}. Washington, dc.
	\item
	Moses (1967) Moses, J. (1967) Symbolic integration. \emph{Report}
	mac-tr-47. Project mac. Cambridge, Mass.: mit.
	\item
	Moses (1971) Moses, J. (1971) Algebraic simplification: a guide for
	the perplexed. \emph{Proc. Second Symp. on Symbolic and Algebraic
		Manipulation}, pp. 282-303. (ed. Petrich, S. R.). New York:
	Association for Computation Machinery.
	\item
	Nevins (1971) Nevins, A. J. (1971) A human-oriented logic for
	automatic theorem proving. tm-62789. George Washington University.
	\item
	Pietrzykowski (1971) Pietrzykowski, T. (1971) A complete mechanism of
	second-order logic. Science Research Report (csrr 2038). Department of
	Analysis and Computer Science, University of Waterloo.
	\item
	Pietrzykowski \& Jensen (1972) Pietrzykowski, T. \& Jensen, D. C.
	(1972) A complete mechanism of $\omega$-order logic. Science
	Research Report csrr 2060. Department of Analysis and Computer
	Science, University of Waterloo.
	\item
	Plotkin (1972) Plotkin, G. D. (1972) Some unification algorithms.
	Research Memorandum (forthcoming). School of Artificial Intelligence,
	University of Edinburgh.
	\item
	Robinson (1965) Robinson, J. A. (1965) A machine-oriented logic based
	on the resolution principle. \emph{J. Ass. comput. Mach.},
	\textbf{12}, 23-41.
	\item
	Rulfson (1970) Rulfson, J. F. (1970) Preliminary specification of the
	qa4 language. Artificial Intelligence Group. Technical Note 50.
	Stanford Research Institute.
	\item
	Schoenfield (1967) Schoenfield, J. R. (1967) \emph{Mathematical
		logic}. Reading, Mass.: Addison-Wesley.
\end{itemize}

\end{document}          
